% Created 2020-04-05 Sun 22:12
% Intended LaTeX compiler: pdflatex
\documentclass{article}
\usepackage[utf8]{inputenc}
\usepackage[T1]{fontenc}
\usepackage{graphicx}
\usepackage{grffile}
\usepackage{longtable}
\usepackage{wrapfig}
\usepackage{rotating}
\usepackage[normalem]{ulem}
\usepackage{amsmath}
\usepackage{textcomp}
\usepackage{amssymb}
\usepackage{capt-of}
\usepackage{hyperref}
\date{\today}
\title{}
\hypersetup{
 pdfauthor={},
 pdftitle={},
 pdfkeywords={},
 pdfsubject={},
 pdfcreator={Emacs 26.3 (Org mode 9.3.6)}, 
 pdflang={English}}
\begin{document}

\tableofcontents

\section{S1: Parsing bitmap represntation of the data}
\label{sec:org5f0527b}
Funcationlity releating to reading/writing to external disk.
\begin{description}
\item{F1} Read the external disk, and set that up as the workign disk.
\begin{description}
\item{I} File that consisent a stream of formated bits.
\item{O} Set up the disk data structure
\item{S1} Read external disk from file
\begin{itemize}
\item Each sector is represtned as a new line.
\item The sector sequence. Repeat sectors 2 and 3.
\begin{enumerate}
\item Super block
\item inodeCount
\item Set of data blcoks, each allociated to each inode.
\end{enumerate}
\item superblock structure
The whole sector is just a bitset, that is the super number.
\item inode Strucutre
\begin{itemize}
\item Since multiple indoes are within a single sector, we have to divide by the number of bits.
\begin{itemize}
\item \(NumberInodes(1 +  \log_2(\text{WOW}) + \log_2(125 * SectorNum))\) This is the general amount of bits used to reprsent each indoe
\item{1} The type of inode it is (file/directory)
\item{\(\log_2(512*SectorSize)\)} The size of the datablock.
\end{itemize}
\end{itemize}
\end{itemize}
\end{description}
\end{description}
\end{document}